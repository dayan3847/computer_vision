
\documentclass[24pt]{article}


\usepackage{cancel}

% =================================
% IDIOMA
\usepackage[utf8]{inputenc}
\usepackage[spanish, mexico]{babel}
% =================================


% ===========================
% GRÁFICOS
\usepackage{geometry}
\geometry{top=1.6cm, bottom=1.8cm, left=2.8cm, right=2.5cm}

\usepackage{tcolorbox}
\tcbuselibrary{breakable}
\tcbuselibrary{theorems}
% ===========================


% ====================
% MATEMÁTICAS
\usepackage{amsmath}
\usepackage{amsthm}
\usepackage{amssymb}
\usepackage{mathtools}
\usepackage[rflt]{floatflt}
\usepackage{parskip}
\usepackage{graphicx}
\usepackage{cancel}
\usepackage{mathrsfs}
% ====================


% ===================
% MACROS ÚTILES
\usepackage{array}
\usepackage{booktabs}
\usepackage{enumitem}
% ===================


% =======================================
% FUENTE
\usepackage[T1]{fontenc}
\usepackage{lmodern}
\renewcommand{\familydefault}{\sfdefault}
% =======================================


% ==========================================================
% CONTADORES
\newcounter{ejercicio}
\newcommand{\numej}{\refstepcounter{ejercicio}\theejercicio}

\newcounter{lema}
\newcommand{\numek}{\refstepcounter{lema}\thelema}
% ==========================================================


% ======================================================================================
% TCOLORBOX
\newtcolorbox{ejer}[1][\numej]{title={Ejercicio #1}, colback=blue!5, colframe=blue!35!black, fonttitle=\bfseries, breakable, parbox=false, before skip= \bigskipamount, after skip = \medskipamount}

\newtcolorbox{ejercont}{colback=blue!5, colframe=blue!50!black, breakable, parbox=false}

\newtcolorbox{lema}[1][\numek]{title={Lema #1.}, colback=green!5, colframe=green!35!black, fonttitle=\bfseries, breakable, parbox=false, before skip= \bigskipamount, after skip = \medskipamount}

\newtcolorbox{lemacont}{colback=green!5, colframe=green!50!black, breakable, parbox=false}
% ======================================================================================


% ==============================================
% DISEÑO
\renewcommand{\labelenumi}{\emph{\alph{enumi}})}
% ==============================================


% =====================================================
% MATH MACROS
\newenvironment{answer}{\begin{proof}[Solución.]\let\qed\relax}{\end{proof}}
% =====================================================


% ==================================
% SÍMBOLOS
\renewcommand{\epsilon}{\varepsilon}
% ==================================


% ===================================================
% NUEVOS COMANDOS
\newcommand{\N}{\mathbb{N}}
\newcommand{\Z}{\mathbb{Z}}
\newcommand{\Q}{\mathbb{Q}}
\newcommand{\R}{\mathbb{R}}
\newcommand{\C}{\mathbb{C}}
\newcommand{\F}{\mathbb{F}}
\newcommand{\U}{\text{U}(2, 1)}
\newcommand{\pu}{\text{PU}(2, 1)}
\newcommand{\pc}{\mathbb{P}^2_\mathbb{C}}
\newcommand{\hc}{\mathbb{H}^2_\mathbb{C}}
\newcommand{\gl}{\text{GL}(3, \mathbb{C})}
\newcommand{\pgl}{\text{PGL}(3, \mathbb{C})}

\newcommand{\abs}[1]{\left| #1 \right|}
\newcommand{\paren}[1]{\left( #1 \right)}
\newcommand{\corch}[1]{\left[ #1 \right]}
\newcommand{\set}[1]{\left\{ #1 \right\}}
\newcommand{\vcord}[1]{\left\langle #1 \right\rangle}
\newcommand{\gen}[1]{\mathrm{gen}\left(#1\right)}
% ===================================================


% ===============================================================================
% MATRICES
\newenvironment{apmatrix}[1]{\left(\begin{array}{@{ }#1@{ }}}{\end{array}\right)}
\newcolumntype{M}{>{$} l <{$}}
% ===============================================================================


% =================================================================
% NUMERACIÓN DE ECUACIONES
\newcommand{\tagthis}{\refstepcounter{equation} \tag{\theequation}}
\newcommand{\eqand}{\qquad\text{y}\qquad}

\renewcommand{\theenumi}{\alph{enumi}}
\renewcommand{\labelenumi}{\theenumi ).}
% =================================================================

\begin{document}

	Dayan Bravo Fraga

	Función:
	\[
		f(x,\mu,\sigma)
		= \frac{
			e^{ -\frac{(x-\mu)^2}{2\sigma^2}}
		}{
			\sigma\sqrt{2\pi}
		}
	\]

	Ingualando las dos funciones
	\[
		f(x,\mu_1,\sigma_1) = f(x,\mu_2,\sigma_2)
	\]

	Sustituir:
	\[
		\frac{
			e^{ -\frac{(x-\mu_1)^2}{2\sigma_1^2}}
		}{
			\sigma_1\sqrt{2\pi}
		}
		=
		\frac{
			e^{ -\frac{(x-\mu_2)^2}{2\sigma_2^2}}
		}{
			\sigma_2\sqrt{2\pi}
		}
	\]

	Simplificar valores constantes en los dos miembros:
	\[
		\frac{
			e^{ -\frac{(x-\mu_1)^2}{2\sigma_1^2}}
		}{
			\sigma_1\cancel{\sqrt{2\pi}}
		}
		=
		\frac{
			e^{ -\frac{(x-\mu_2)^2}{2\sigma_2^2}}
		}{
			\sigma_2\cancel{\sqrt{2\pi}}
		}
	\]

	Eliminar valores simplificados:
	\[
		\frac{
			e^{ -\frac{(x-\mu_1)^2}{2\sigma_1^2}}
		}{
			\sigma_1
		}
		=
		\frac{
			e^{ -\frac{(x-\mu_2)^2}{2\sigma_2^2}}
		}{
			\sigma_2
		}
	\]

	Reorganizar:
	\[
		\frac{
			e^{ -\frac{(x-\mu_1)^2}{2\sigma_1^2}}
		}{
			e^{ -\frac{(x-\mu_2)^2}{2\sigma_2^2}}
		}
		=
		\frac{\sigma_2}{\sigma_1}
	\]

	Aplicar propiedades de las potencias:
	\[
		e^{ -\frac{(x-\mu_1)^2}{2\sigma_1^2}+\frac{(x-\mu_2)^2}{2\sigma_2^2}}
		=
		\frac{\sigma_2}{\sigma_1}
	\]

	Aplicar logaritmo natural a ambos miembros:
	\[
		-\frac{(x-\mu_1)^2}{2\sigma_1^2}+\frac{(x-\mu_2)^2}{2\sigma_2^2}
		=
		\ln\frac{\sigma_2}{\sigma_1}
	\]

	Multiplicar ambos miembros por $2\sigma_1^2\sigma_2^2$:
	\[
		2\sigma_1^2\sigma_2^2\paren{-\frac{(x-\mu_1)^2}{2\sigma_1^2}+\frac{(x-\mu_2)^2}{2\sigma_2^2}}
		=
		2\sigma_1^2\sigma_2^2\ln\frac{\sigma_2}{\sigma_1}
	\]

	Efectuar la multiplicación paso 1:
	\[
		-\sigma_2^2(x-\mu_1)^2 + \sigma_1^2(x-\mu_2)^2
		=
		2\sigma_1^2\sigma_2^2\ln\frac{\sigma_2}{\sigma_1}
	\]

	Efectuar la multiplicación paso 2:
	\[
		-\sigma_2^2 \paren{ x^2 -2\mu_1x +\mu_1^2} + \sigma_1^2  \paren{ x^2 -2\mu_2x +\mu_2^2}
		=
		2\sigma_1^2\sigma_2^2\ln\frac{\sigma_2}{\sigma_1}
	\]

	Efectuar la multiplicación paso 3:
	\[
		-  \sigma_2^2 x^2  +2\mu_1\sigma_2^2 x  -\mu_1^2\sigma_2^2
		+ \sigma_1^2 x^2  -2\mu_2 \sigma_1^2 x +\mu_2^2 \sigma_1^2
		=
		2\sigma_1^2\sigma_2^2\ln\frac{\sigma_2}{\sigma_1}
	\]

	Pasar todo para un miembro y ordenar:
	\[
		\sigma_1^2 x^2
		-  \sigma_2^2 x^2
		+2\mu_1\sigma_2^2 x
		-2\mu_2 \sigma_1^2 x
		+\mu_2^2 \sigma_1^2
		-\mu_1^2\sigma_2^2
		-2\sigma_1^2\sigma_2^2\ln\frac{\sigma_2}{\sigma_1}
		=0
	\]

	Agrupar términos semejantes:
	\[
		(\sigma_1^2-  \sigma_2^2) x^2
		+
		(2\mu_1\sigma_2^2-2\mu_2 \sigma_1^2 )x
		+
		\paren{
			\mu_2^2 \sigma_1^2
			-\mu_1^2\sigma_2^2
			-2\sigma_1^2\sigma_2^2\ln\frac{\sigma_2}{\sigma_1}
		}
		=0
	\]

	Extraemos $a$, $b$ y $c$ de la forma $ax^2+bx+c=0$:

	$a=\sigma_1^2-  \sigma_2^2$

	$b=2\mu_1\sigma_2^2-2\mu_2 \sigma_1^2$

	$c=\mu_2^2 \sigma_1^2 -\mu_1^2\sigma_2^2-2\sigma_1^2\sigma_2^2\ln\frac{\sigma_2}{\sigma_1}$


\end{document}