\documentclass[12pt]{article}

% Paquetes necesarios
\usepackage[spanish]{babel}
\usepackage[utf8]{inputenc}
\usepackage[T1]{fontenc}
\usepackage{graphicx}
\usepackage{amsmath}
\usepackage{float}
\usepackage{caption}
\usepackage{subcaption}

% Configuración de márgenes
\usepackage[a4paper, total={6in, 8in}]{geometry}

% Título del documento
\title{Título del reporte}
\author{Dayan Bravo Fraga}
\date{Fecha}

\begin{document}

% Cover of the report
    \begin{titlepage}
        \begin{center}
            \vspace*{1cm}

            \textbf{\LARGE Título del reporte}

            \vspace{0.5cm}
            Subtítulo (opcional)

            \vspace{1.5cm}

            \textbf{Autor}

            \vfill

            \vspace{0.8cm}

%            \includegraphics[width=0.4\textwidth]{logo_uady.jpeg}

            Facultad de Matemáticas\\
            Universidad Autónoma de Yucatán\\
            Mérida, Yucatán, México\\
            marzo de 2023

        \end{center}
    \end{titlepage}

% Índice
    \tableofcontents

% Sección de introducción


    \section{Introducción}
    En esta sección se introducirá el problema y se explicará la motivación detrás del proyecto.

% Sección de marco teórico


    \section{Marco teórico}
    En esta sección se describirán los conceptos teóricos necesarios para entender el proyecto.

% Sección de metodología


    \section{Metodología}
    En esta sección se explicará la metodología utilizada para abordar el problema.

% Sección de resultados


    \section{Resultados}
    En esta sección se presentarán y discutirán los resultados obtenidos.

% Sección de conclusiones


    \section{Conclusiones}
    En esta sección se resumirán las principales conclusiones del proyecto.

% Bibliografía
    \begin{thebibliography}{9}
        \bibitem{ejemplo1}
        Autor. \textit{Título del artículo}. Nombre de la revista. Año.

        \bibitem{ejemplo2}
        Autor. \textit{Título del libro}. Editorial. Año.
    \end{thebibliography}

\end{document}
