%proof
\section{Proof}\label{sec:proof}

\subsection{Imagen de 3x3}\label{subsec:imagen-de-3x3}

Realizaremos una pequeña prueba con una imagen de 3x3, para ver que el algoritmo funciona correctamente.

La Figura~\ref{fig:test_img_3x3} se va a utilizar para probar el algoritmo.

\begin{figure}
    \centering
    \includegraphics[width=0.3\textwidth]{./latex/img/original}
    \caption{\label{fig:test_img_3x3}Imagen de prueba de 3x3.}
\end{figure}

Al inicializar la máscara, todos los píxeles pertenecen a la misma clase, por lo que la máscara se inicializa con todos los píxeles con el valor 0.
La máscara es la siguiente matriz:

\begin{equation*}
    \begin{bmatrix}
        0 & 0 & 0 \\
        0 & 0 & 0 \\
        0 & 0 & 0 \\
    \end{bmatrix}
\end{equation*}

Si graficamos la imagen con los valores promedio de todos los píxeles (de esa clase), obtenemos la siguiente imagen:

En una tabla lo podemos ver de la siguiente forma:


\begin{table}
    \centering
    \begin{tabular}{|c|c|}
        \hline
        Mask & Image (L=0) \\
        \hline
        \begin{equation*}
            \begin{bmatrix}
                0 & 0 & 0 \\
                0 & 0 & 0 \\
                0 & 0 & 0 \\
            \end{bmatrix}
        \end{equation*}
        &
        \includegraphics[width=0.25\textwidth]{./latex/img/m0}
        \\
        \hline
        \begin{equation*}
            \begin{bmatrix}
                2 & 2 & 1 \\
                2 & 1 & 2 \\
                2 & 2 & 1 \\
            \end{bmatrix}
        \end{equation*}
        &
        \includegraphics[width=0.25\textwidth]{./latex/img/m1}
        \\
        \hline
        \begin{equation*}
            \begin{bmatrix}
                5 & 6 & 3 \\
                6 & 4 & 6 \\
                5 & 5 & 3 \\
            \end{bmatrix}
        \end{equation*}
        &
        \includegraphics[width=0.25\textwidth]{./latex/img/m2}
        \\
        \hline
        \begin{equation*}
            \begin{bmatrix}
                9  & 11 & 7  \\
                12 & 4  & 11 \\
                10 & 9  & 8  \\
            \end{bmatrix}
        \end{equation*}
        &
        \includegraphics[width=0.25\textwidth]{./latex/img/m3}
        \\
        \hline
        \begin{equation*}
            \begin{bmatrix}
                13 & 15 & 7  \\
                12 & 4  & 16 \\
                10 & 14 & 8  \\
            \end{bmatrix}
        \end{equation*}
        &
        \includegraphics[width=0.25\textwidth]{./latex/img/m4}
        \\
        \hline
    \end{tabular}
    \caption{\label{tab:widgets}An example table.}
\end{table}

\newpage

Como podemos observar, la imagen tiene todos los píxeles con el mismo valor (promedio de todos los píxeles)

Ahora podemos dividir la clase 0 en 2 clases, luego se realiza la umbralización y se obtiene la siguiente máscara:

Si graficamos la imagen con los valores promedio de los píxeles de las nuevas clases (1 y 2), obtenemos la siguiente imagen:

Como podemos observar, la imagen se dividió en 2 clases, una con el valor promedio de los píxeles de la clase 1 y otra con el valor promedio de los píxeles de la clase 2.
